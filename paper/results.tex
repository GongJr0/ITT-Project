\section{Results}\label{sec:results}

The first course of action is the verification of \hyperref[hyp]{H1}.
Therefore, we compare the aGVA observations of all quantile levels for exporters and non-exporters.

\begin{table}[H]
    \centering
    \label{fig:ks_table_all_firms}
    \begin{tabular}{lcc}
        \toprule
        $Q_i$ & $p$-value & Equality Conclusion \\
        \midrule
        $Q_{10}$ & $\approx 0$ & Reject $H_0$ \\
        $Q_{25}$ & $\approx 0$ & Reject $H_0$ \\
        $Q_{50}$ & $\approx 0$ & Reject $H_0$ \\
        $Q_{75}$ & $\approx 0$ & Reject $H_0$ \\
        $Q_{90}$ & $\approx 0$ & Reject $H_0$ \\
        \bottomrule
    \end{tabular}
    \caption{KS Test Results: Exporters vs. Non-Exporters (All Firms)}
\end{table}

\noindent As expected, and demonstrated in literature, we observe a strong statistical evidence of productivity difference in the UK market.
However, seeing a significant difference with all categories included does not guarantee a difference will be observed when sub-categories are tested separately.

\vspace{\baselineskip}

We then explore the productivity differences between domestic and foreign-owned (not necessarily multinational) firms' export status.
For all quantiles, foreign/domestic owned exporters and non-exporters are compared for equality.
\begin{table}[H]
    \centering
    \textbf{p-values for Exporters vs. Non-Exporters by Ownership Type}
    \begin{tabular}{lll}
\toprule
 & Foreign-Owned & Domestic-Owned \\
\midrule
p10 & $\approxeq 0$ & $\approxeq 0$ \\
p25 & $\approxeq 0$ & $\approxeq 0$ \\
p50 & $\approxeq 0$ & $\approxeq 0$ \\
p75 & 0.015643 & $\approxeq 0$ \\
p90 & 0.594071 & $\approxeq 0$ \\
\bottomrule
\end{tabular}

    \caption{KS Test Results: Exporters vs. Non-Exporters by Ownership Type}
\end{table}

