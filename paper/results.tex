\section{Results}\label{sec:results}

The first course of action is the verification of \hyperref[hyp]{H1}.
Therefore, we compare the aGVA observations of all quantile levels for exporters and non-exporters.

\begin{table}[H]
    \centering
    \label{fig:ks_table_all_firms}
    \begin{tabular}{lcc}
        \toprule
        $Q_i$ & $p$-value & Equality Conclusion \\
        \midrule
        $Q_{10}$ & $\approxeq 0$ & Reject $H_0$ \\
        $Q_{25}$ & $\approxeq 0$ & Reject $H_0$ \\
        $Q_{50}$ & $\approxeq 0$ & Reject $H_0$ \\
        $Q_{75}$ & $\approxeq 0$ & Reject $H_0$ \\
        $Q_{90}$ & $\approxeq 0$ & Reject $H_0$ \\
        \bottomrule
    \end{tabular}
    \caption{KS Test Results: Exporters vs. Non-Exporters (All Firms)}
\end{table}

\noindent As expected, and demonstrated in literature, we observe a strong statistical evidence of productivity difference in the UK market.
However, seeing a significant difference with all categories included does not guarantee a difference will be observed when sub-categories are tested separately.

\vspace{\baselineskip}

We then explore the productivity differences between domestic and foreign-owned (not necessarily multinational) firms' export status.
For all quantiles, foreign/domestic owned exporters and non-exporters are compared for equality.
\begin{table}[H]
    \centering
    \textbf{p-values for Exporters vs. Non-Exporters by Ownership Type}
    \begin{tabular}{lll}
\toprule
 & Foreign-Owned & Domestic-Owned \\
\midrule
p10 & $\approxeq 0$ & $\approxeq 0$ \\
p25 & $\approxeq 0$ & $\approxeq 0$ \\
p50 & $\approxeq 0$ & $\approxeq 0$ \\
p75 & 0.015643 & $\approxeq 0$ \\
p90 & 0.594071 & $\approxeq 0$ \\
\bottomrule
\end{tabular}

    \caption{KS Test Results: Exporters vs. Non-Exporters by Ownership Type}
\end{table}

\noindent Although most tests result in a strong rejection, we see an interesting result in the foreign-owned firms.
As we cross $Q_{75}$, the $p$-values start increasing ultimately leading to a $Q_{90}$ where we fail to reject the equality of distributions.
In the UK data, we can therefore conclude that foreign-owned companies at the upper echelon of labor productivity do not exhitibit an increased productivity tied to their export status.

\vspace{\baselineskip}

Finally, we test \hyperref[hyp]{H3} by comparing exporters/non-exporters of each size band at all quantile levels.
Under H3, we expect to see larger $p$-values as we move towards larger size bands and higher quantile brackets.

\begin{table}[H]
    \centering
    \textbf{p-values for Exporters vs. Non-Exporters by Size Band}
    \begin{tabular}{lcccc}
\toprule
 & Micro (0-9) & Small (10-49) & Medium (50-249) & Large (250+) \\
\midrule
$Q_{10}$ & $0.001116^*$ & $\approxeq 0$ & $\approxeq 0$ & $\approxeq 0^*$ \\
$Q_{25}$ & $\approxeq 0^*$ & $\approxeq 0$ & $\approxeq 0$ & $\approxeq 0^*$ \\
$Q_{50}$ & $\approxeq 0^*$ & $\approxeq 0$ & $\approxeq 0$ & $\approxeq 0^*$ \\
$Q_{75}$ & $\approxeq 0^*$ & $\approxeq 0$ & $\approxeq 0$ & $\approxeq 0^*$ \\
$Q_{90}$ & $\approxeq 0^*$ & $\approxeq 0$ & $\approxeq 0$ & $\approxeq 0^*$ \\
\bottomrule
\end{tabular}

    \caption{KS Test Results: Exporters vs. Non-Exporters by Size Band}
\end{table}

\noindent Contrary to our expectations, we saw a strict and strong rejection in all sizes and quantile levels.
This shows that the positive relationship between export status and labor productivity is persistent across all sizes of firms in the UK market.

\vspace{\baselineskip}
Seeing the result of the H3 tests, we further explored the relationship between size and productivity within each export status.
We compared the size bands within exporters and non-exporters separately to see if growth in size leads to statistically significant increases in productivity.

\begin{table}[H]
    \small
    \centering
    \textbf{p-values for Size Band Comparisons within Exporters}
    \begin{tabular}{lcccc}
\toprule
& Micro (0-9) & Small (10-49) & Medium (50-249) & Large (250+) \\
\midrule
Micro (0-9) & 1 & $\approxeq 0$  & $\approxeq 0$ & $\approxeq 0^*$ \\
Small (10-49) & $\approxeq 0$  & 1 & 0.392945 & 0.001347 \\
Medium (50-249) & $\approxeq 0$  & 0.392945 & 1 & 0.034520 \\
Large (250+) & $\approxeq 0^*$  & 0.001347 & 0.034520 & 1 \\
\bottomrule
\end{tabular}

\end{table}

\begin{table}[H]
    \small
    \centering
    \textbf{p-values for Size Band Comparisons within Non-Exporters}
    \begin{tabular}{llll}
\toprule
 & Small (10-49) & Medium (50-249) & Large (250+) \\
\midrule
Small (10-49) & 1 & 0.000001 & 0.678914 \\
Medium (50-249) & 0.000001 & 1 & 0.000055 \\
Large (250+) & 0.678914 & 0.000055 & 1 \\
\bottomrule
\end{tabular}

    \caption{KS Test Results: Size Band Comparisons within Exporters and Non-Exporters.}
\end{table}
{\tiny $^*$ Both samples being compared have small sample sizez (< 20). The $p$-value is included but should not be treated as accurate.}

\noindent We observe that most size combinations lead to a rejection