\section{Conclusion}\label{sec:conclusion}

\noindent Through-out this study, our aim was to investigate the results of the framework introduced by Melitz. 
We believe that our findings provide beneficial insights into the casual relationship between firm productivity and international trade.
Our results verifies the self selection hypothesis of Meliz by showing that labor productivity distribution of exporters is significantly different from that of non-exporters across all levels of quantiles.
Moreover, we find that within exporters and nonexporter, firm-size plays a cruical role in determining the labor productivity distribution of firms. 
Albeit, we don't put forward a conclsuive evidence such as elasticity estimates.
Further research can be conducted to estimate the magnitue of the effects in more depth by employing different methodologies.
Additionally, our study is limited to the UK market due to data availability.
Future studies can be carried out on firm level data from different countries to see whether the findings hold true in different economic contexts.
Overall, we find it sufficient to conclude the differnece of productivty setting between exporters and nonexports in simple yet effective manner.

