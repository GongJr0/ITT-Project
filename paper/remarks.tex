\section{Remarks}\label{sec:remarks}
\subsection{On the application of the KS Test}\label{subsec:ks_remarks}
Our data has two properties that creates potential violations of the KS assumptions:
\begin{itemize}
    \item \textbf{Dependence within Samples}: Having time-dependent observations within each sample, it is fair to expect some level of autocorrelation in-sample. 
    If such autocorrelation exists, our samples would no longer be \textbf{i.i.d.} thus rendering the KS test invalid. 
    \item \textbf{Dependence between Samples}: Our samples are drawn from the same time periods, and slow-moving trends act on productivity in the same ways for exporters and non-exporters. 
    For example, technological advancements ($A$) increases labor productivity for both exporters and non-exporters over time. In a shared inter-sample macroeconomic environment,
    there will likely be some level of correlation between the samples as well. As well as putting the i.i.d. assumption on shaky grounds, this has the potential to create inter-sample interactions that the KS test is not designed to account for. 
\end{itemize}

To assure brevity and remaining within the page count limitations, we elected to assume that the KS test is applicable to all samples evaluated. We understand that this would very likely be undefensible as an empirical analysis methodology.
With no such limitations, could have elected to define a Block Bootstrap KS test or move further with alternative non-parametric tests.