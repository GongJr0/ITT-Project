\section{Remarks}\label{sec:remarks}
\subsection{Regarding Time-Dependent Changes in Firm Structure}

Evaluating the productivity impact of export activity on a given firm requires observations from before and after the firm has been informed of its export status.
Therefore, the observation period will potentially include expansionary or contractionary actions taken over time.
For example, a firm may choose to expand in headcount or capital stock given the projected demand increase brought by an export opportunity.

\vspace{\baselineskip}

Due to the possibility of such changes, any fluctuations in productivity cannot be solely attributed to the premise of increased exports.
For the purposes of this study, we assume that changes not directly related to export status are minimal and slow-moving relative to the export-related changes.
