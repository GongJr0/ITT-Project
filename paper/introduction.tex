\section{Introduction}\label{sec:introduction}


\subsection{Overview}\label{subsec:overview}
Benefits of the trade has been widely studied in the literature and every model that has been developed has its own take on how trade benefits between countries. 
However, from simple to complex models, there was a one common theme which attracted researchers attention called "productivity". 
In the simple models the term productivity was implied through the concept of comparative advantage. 
Which basically supports the idea that countries have different productivity levels in producing different goods. 
Countries which have lower opportunity cost for a certain type of good --often measured by hours of labor required to produce one unit of good-- will specialize in producing that good and trade with other countries (Ricardo, 1817). 
Subsequent models have built on different aspects of productivity but one of them was groundbreaking in terms of its implications. Melitz (2003) introduced a whole new methodology to assess the productivity. 
He focused on microdata of firms and introduced the concept of firm-level productivity. 
According to Melitz (2003), firms have different productivity levels within the same industry. 
Some firms are more productive than others and this artifact come with some important consequences. 
First, would be that only the more productive firms will be able to cover the fixed costs of entering the export market as they would be able to generate enough revenue to cover those costs. 
Second, less prodcutive firms would not be able to cover those costs and would be forced to exit the market. 
As a result, average productivity of the industry would increase since only the more productive firms remanin in the market. 
With this in mind , in our study, we tried to assess the the outcomes of the Melitz (2003) model by comparing the labor productivity of different type of firms (i.e. exporters vs. non-exporters, multinational vs. domestic etc.) using firm-level data from Office of National Statistics (ONS) of UK.


\subsection{Hypotheses}\label{subsec:hypotheses}

We test the following hypothesis in our study:
\begin{itemize}
    \item H1: Exporting firms labor productivity distribution is not same as non-exporting firms.
    \item H2: Multinational firms labor productivity distribution is not same as domestic firms.
    \item H3: Labor productivity distribution of exporting firms could be dependent on firm size. Some firm size categories could have same labor productivity distribution
    \end{itemize}


\subsection{Literature Review}\label{subsec:lit_review}
There are numerous studies that focuses on the relationship between productivity and export status of firms. 
Empirical work carried out by Bernard and Jensen (1995) is one of the remarkable studies in the field investigating the relation between export status and productivity.
They use plant level data from U.S. manufacturing plants between 1976 and 1987 to compare the productivity of exporting and non-exporting plants.
Their findings suggest that exporting plants are more productive than non-exporting plants, which aligns with the self-selection hypothesis proposed by Melitz (2003).
They also define the relationship one sided. Good firms likely to become expoerters but exporting itself does not necessarily increase productivity. Also Melitz (2003) has the same view.
Another interesting study has been published by Atkin, Khandelwal, and Osman (2017) they conduct a survey among rug producers in Egypt. 
Firms participating to the experiment divides into two groups : control and treatment. 
Nothing has been told to the treatment group prior to the experiment as same as the control group. 
An intermediary rug firm in Egypt used to find buyers in the international market and only the first orders are initiated by the intermediary the rest is dependent on the satisfaction of the customer and the quality of the rugs produced. 
Over time, they record the improvements of the treatment group based on the quality of the rugs produced. By doing that, they capture the learning by doing effect of exporting in the treatment groups. 
At the end of their experiment they find that the profits of the treatment group increases by 16\% to 26\% compared to the control group along with the quality improvements of the rugs produced by the treatment group. 
What makes this study more roboust is that at the end of the experiment they gather the control and treatment group and provide them the necessary materials to weave a rug. 
Treatment group produces much higer quality rugs compared to the control group at the same production time. This shows that exporting has positive effect on treatment groups. 
Further validation is valuable as profit measures could be biased due to certain affects like product specialization.


