\section{Introduction}\label{sec:introduction}


\subsection{Overview}\label{subsec:overview}
International trade relations has been proven to be mutually beneficial over the centuries of economic history. 
Macroeconomic models have constantly been constructed to reward trade, and countless studies have been published to validate those models empirically.
A variable of interest that constantly appeared in simple to complex models was \textbf{labor productivity}.
Productivity has its roots in the concept of comparative advantage introduced by David Ricardo in 1817.
Ricardo's theory of comparative advantage suggested that countries have varying levels of productivity in producing different goods.
Countries which have lower opportunity cost for a certain type of good \textendash~often measured by hours of labor required to produce one unit of good \textendash~will specialize in producing that good and trade with other countries. 
Subsequent models have built on different aspects of productivity but one of them was groundbreaking in terms of its implications. (\cite{melitz_impact_2003}) 
Melitz focused on micro data of firms and introduced the concept of firm-level productivity. 
According to Melitz, firms have different productivity levels within the same industry. 
Some firms are more productive than others and this property brings with it some important consequences.

\vspace{\baselineskip} 

Given all agents are rational, the model suggests that comparatively productive firms will be able to cover the fixed costs of entering the export market through efficient generation of revenue. 
Conversely, firms with lesser productivity would not be able to cover those costs and would be forced to exit the international market. 
As a result, the average productivity of industries increase as only the more productive firms can afford to operate at an international scale. 
% With this in mind, in our study, we tried to assess the outcomes of the Melitz (2003) model by comparing the labor productivity of different type of firms (i.e. exporters vs. non-exporters, multinational vs. domestic etc.) using firm-level data from Office of National Statistics (ONS) of UK.
With a focus on the UK market, our study aims to explore the interactions between international trade and labor productivity in a granular manner and test the theoretical expectations brought by Melitz.

\subsection{Hypotheses}\label{subsec:hypotheses}
Effects of international trade on labor productivity may differ for different subgroups of firms. 
Although an overall positive relationship has long been agreed upon, the extent/existence of this relationship can be explored further by creating sub-categories of firms.
As such, we formed multiple hypotheses to understand empirical presence of the Melitz framework in the UK market. 

\vspace{2\baselineskip}

\noindent We test the following hypothesis in our study:
\begin{itemize}
    \phantomsection
    \label{hyp}
    \item \textbf{H1:} Labor productivity of exporters show statistically significant differences from that of non-exporters in all levels of productivity.
    \item \textbf{H2:} Multinational firms' labor productivity distribution is not same as domestic firms for both exporters and non-exporters.
    \item \textbf{H3:} As firms grow in size, the gap in labor productivity between exporters and non-exporters tends to decrease as larger firms can utilize economies of scale regardless of their market scope.
    By extension, the labor productivity distribution of exporters and non-exporters approach the same limit for sufficiently large firms.
\end{itemize}

\noindent H1 aims to validate the core premise of the Melitz framework while H2 and H3 are extensions that explore whether Melitz's ideas are unanimously applicable across different firm characteristics. 


\subsection{Literature Review}\label{subsec:lit_review}
There are numerous studies that focus on the relationship between productivity and export status of firms. 
Empirical work carried out by Bernard and Jensen is one of the remarkable studies in the field investigating the relation between export status and productivity.
They use plant-level data from U.S. based manufacturers between 1976 and 1987 to compare the productivity of exporting and non-exporting plants.
The study concludes exporting plants to be more productive than non-exporting plants (\cite{bernard_exceptional_1999}), which aligns with the self-selection hypothesis proposed by Melitz.
They also uncover a one-sided causal reationship where productive firms are likely to become expoerters but exporting itself does not necessarily increase productivity. 
This outcome is aligned with the expectations of Melitz model as well.

\vspace{\baselineskip}

Another interesting study has been published by Atkin, Khandelwal, and Osman where a survey among rug producers in Egypt is conducted. 
Firms participating the experiment are divided into two groups:

\begin{itemize}
    \item Control Group: Firms that do not receive any assistance to export their products and operate at a domestic scale.
    \item Treatment Group: Firms that are assisted to export their products internationally through an intermediary.
\end{itemize}

The experiments starts with both groups being unaware of the export opportunities that will be presented to the treatment group.
An intermediary rug firm in Egypt is used to find buyers in the international market and only the first orders are initiated by the intermediary while the rest is dependent on the satisfaction of the customer and the quality of the rugs produced. 
Over time, they record an improvement in rug quality produced by the treatment group. 
By extension, they capture the \enquote{learning by doing}\footnote{The terms refers to the fact that producers gain experience and optimize their processes over time.} effect of exporting through a measurable difference in quality between the treatment pre/post export opportunity. 
At the end of their experiment they find that the profits of the treatment group increases by 16\% to 26\% compared to the control group along with the quality improvements of the rugs produced by the treatment group. 
What makes this study more robust is that at the end of the experiment they gather the control and treatment group and provide them the necessary materials to weave a rug. 
Treatment group produces much higer quality rugs compared to the control group at the same production time. This shows that exporting has positive effect on treatment groups. 
Further validation is valuable as profit measures could be biased due to certain affects like product specialization.


