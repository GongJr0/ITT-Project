\section{Data}\label{sec:data}
\noindent This study utilizes a \href{https://www.ons.gov.uk/economy/economicoutputandproductivity/productivitymeasures/datasets/tradeandproductivityingreatbritainevidencefromfirmlevelmicrodata}{dataset} of 1,800 entries published by the \enquote{Office for National Statistics} (ONS) of the United Kingdom.
The dataset focuses on labor productivity through \textbf{\enquote{Approximate Gross Value Added}} (aGVA) per worker.
A granular set of subcategories are provided alongside the ability to easily filter exporters and non-exporters.

\vspace{\baselineskip}

\noindent Unfortunately, the dataset is not recorded at the firm level.
However, the granularity of segmentation create easily accessible and appropriately generalized productivity measures that can be compared directly.
Entries follow an annual frequency, spanning from 2005 to 2022.\footnote{The data for 2005-2011 comes from a single source (ABS-TiG) while every data point from 2011 onwards have two separate sources (ABS-TiG and ABS) with entries for each source kept separate.}
\vspace{\baselineskip}

\noindent The data measures quantiles $\hat{Q} = \lbrace Q_{10}, Q_{25}, Q_{50}, Q_{75}, Q_{90}\rbrace$ alongside the mean $\big(\, \overline{aGVA}\, \big)$ and the standard deviation ($S_{aGVA}$).
Said measurements are provided for exporters and non-exporters separately.
Observations are further segmented by:
\begin{itemize}
    \item \textbf{Size Bands (Employees)}: Small (10-49), Medium (50-249), Large (250+)
    \item \textbf{Age Bands (Years)}: New (0-2), Young (3-5), Mature (6-10), Old (10+)
    \item \textbf{Ownership Type}: Domestic, Foreign
    \item \textbf{Industry}: NACE Rev. 2 Categories
    \item \textbf{Region}: 11 UK regions specified by ONS
\end{itemize}